% Disciple-BD.tex
\begin{hcarentry}[section]{Disciple}
\report{Ben Lippmeier}%05/13
\status{experimental, active development}
\participants{Ben Lippmeier, Amos Robinson, Erik de Castro Lopo, Kyle van Berendonck}
\makeheader

The Disciplined Disciple Compiler (DDC) is a research compiler used to investigate program transformation in the presence of computational effects. It compiles a family of strict functional core languages and supports region, effect and closure typing. This extra information provides a handle on the operational behaviour of code that isn't available in other languages. Programs can be written in either a pure/functional or effectful/imperative style, and one of our goals is to provide both styles coherently in the same language.

\WhatsNew

DDC is in an experimental, pre-alpha state, though parts of it do work. In March this year we released DDC 0.4.1, with the following new features:

\begin{itemize}
\item Added a bi-directional type inferencer based on Joshua Dunfield and Neelakantan Krishnaswami's recent ICFP paper.
\item Added a region extension language construct, and coeffect system.
\item Added the Disciple Tetra language which includes infix operators and desugars into Disciple Core Tetra.
\item Compilation of Tetra and Core Tetra programs to C and LLVM.
\item Early support for rate inference in Core Flow.
\item Flow fusion now generates vector primops for maps and folds.
\item Support for user-defined algebraic data types.
\item Civilized error messages for unsupported or incomplete features.
\item Most type error messages now give source locations.
\item Building on Windows platforms.
\item Better support for foreign imported types and values.
\item Changed to Git for version control.
\end{itemize}

\FurtherReading
  \url{http://disciple.ouroborus.net}
\end{hcarentry}
