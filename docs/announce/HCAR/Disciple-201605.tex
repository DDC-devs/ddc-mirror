% Disciple-BD.tex
\begin{hcarentry}[section]{Disciple}
\report{Ben Lippmeier}%05/14
\status{experimental, active development}
\participants{
        Amos Robinson, Max Swadling, Kyle Van Berendonck, Jacob Stanley,
        Viktar Basharymau, Erik de Castro Lopo, Ben Lippmeier}
\makeheader

The Disciplined Disciple Compiler (DDC) is a research compiler used to investigate program transformation in the presence of computational effects. It compiles a family of strict functional core languages and supports region and effect typing. This extra information provides a handle on the operational behaviour of code that isn't available in other languages. Programs can be written in either a pure/functional or effectful/imperative style, and one of our goals is to provide both styles coherently in the same language.

\WhatsNew

DDC is in an experimental, pre-alpha state, though parts of it do work. In March this year we released DDC 0.4.2, with the following new features:

\begin{compactitem}
\item Added LLVM code generation for higher order functions.
\item Added automatic insert of run and box casts.
\item Added multi-module compilation.
\item Added desugaring of guards.
\item Added primitive for working with arrays of boxed values and vector of primitive unboxed values.
\item Added first cut PHP code generator.
\item Added case-of-known-constructor transform.
\item Added clustering and rate inference for Core Flow language.
\item Source programs now accept unicode lambdas and dumps of intermediate code use lambdas
      for both term and type binders.
\item Removed deprecated Eval and Lite language fragments.
\end{compactitem}

\FurtherReading
  \url{http://disciple.ouroborus.net}
\end{hcarentry}
